%!TEX root = main.tex
 
\section*{Changelog}

%% -----------------------------------------------------------------------------
\subsection*{Version 1.0}
Version for first review

\begin{itemize}
	\item introduced the argumentation about the crossfeed ratio parametrization in Sec.\ref{2DtotalFit}
	\item updated \cref{fig:stream0_Total2Dfit_charged_corrLambdaC} in Sec.\ref{2DtotalFit}, Table \ref{tab:SixStreams_chargedCorrLam2Dfits}, \cref{fig:RecoSignal_fit-expectedPlot} and \cref{fig:charged_corrLambdaRecoSignal_deviations} (adjusting the comments)
	\item changed the linearity tests plots for charged correlated decays ( \cref{fig:LinearityTest_chargedCorrLambdaC} - \cref{fig:LinearityTest_BR_chargedCorrLambdaC} and \cref{fig:Charged_anticorrLambda_LinearityTest} and \cref{fig:Charged_anticorrLambda_BR_LinearityTest} for anticorrelated decays)
	\item updated the systematics for chargeed correlated decays: summary Table \ref{tab:systematics:ChargedCorr},   \cref{sec:chargedCorrCrossfeedPDF}  updated with the results from the 2D fit (having the crossfeed ratio param.), 
same for \cref{sec:chargedCorrCrossfeedSys}. And for charged anticorrelated decays: summary \cref{tab:systematics_ChargedAnticorr} and Sections 6.8 -6.9.
	\item added the section about the systematics deriving from the parametrization of crossfeed normalization in  the 2D fit (\cref{sec:CrossBkgNormalization} and in charged anticorrelated decays \cref{sec:chargedAnticorrCrossBkgNormalization} ), which takes into account the statistical uncertainties of the parameters.
	\item added the sections about the crossfeed peaking fraction in the 2D fit for anticorrelated decays (\cref{sec:PeakingCrossBkg} and \cref{sec:chargedAnticorrPeakingCrossBkg}) 
	\item  Updated \cref{tab:SixStreams_chargedAnticorrLam2Dfits} for anticorrelated decays and also the corresponding plots.
	%\item changed also for the anticorrelated decays the lienarity test plots
	\item updated \cref{tab:SixStreams_chargedAnticorrLamBR} for BR values of charged anticorrelated decays
	\item in the control sample chapter, updated Section 5.6 for the 2D fit on data, just adding the 2D fit performed on data using the parametrized normalization of crossfeed
background, with results. And in the last section \cref{sec:chargedControlBRvalues} added the new BR measured value for data.
    \item added Tracking efficiency to the systematics (see Sections 4.15 - 6.14)
    \item updated  \cref{tab:chargedControlSyst} for systematics on the control decay
    \item added Figures \ref{fig:chargedBtoD_FOMvsR2_cut} , \ref{fig:chargedBtoD_FOMvsSigProb_cut} and \ref{fig:chargedcorrD0_Pcms} in Appendix \ref{chargedBtoD0App} relative to the optimized cuts discussed in \cref{Sec:SigSelectionOpt}
    \item corrected \cref{fig:chargedBcorr_CrossfeedNoLambdaCpeak}
    %\item added argumentation on using the FEI efficiencies ratios
    %\item added Figures \ref{fig:chargedBcorr_Crossfeed},  \ref{fig:chargedBcorr_CrossfeedLambdaCpeak} and \ref{fig:chargedBcorr_CrossfeedNoLambdaCpeak} to compare the Mbc distributions of events with/without peaking $\Lambda_c$.
   % \item added Fig. \subref{fig:off-resData_charged_corrLambdaC_InvM_woCS} for  $M(p K \pi)$ w/wo continuum suppression comparison 
    %\item description of continuum background modeling in Sec. \ref{sec:2DpdfChargedCorrBtoLambdaC} made more comprehensible
    %\item moved toyMC plots on page 29 to Sec.\ref{2DtotalFit} with comment about the pulls.
    %\item added some comments about Fig. \ref{fig:stream0_chargedBtag_Total_Signal_fit_restrictedRange} - Fig. \ref{fig:NeutralCrossfeed_stream0_corrLambdaC_chargedBtagFit}
    %\item added PID correction section in Chap. \ref{sec:chargedCorrBtoLambdaC}
    %\item added Sec. \ref{sec:corrDataSidebandFit} about the data sideband fit and qaulity of the shapes description.
    %\item added \cref{tab:SixStreams_chargedCorrLamBR}
    
 
\end{itemize}

